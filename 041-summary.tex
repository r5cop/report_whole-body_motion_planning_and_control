Whole-body control is a promising control method for domestic service robots. However, since whole-body control is a potential field-based method, it only operates at a local scope and it might therefore be unable to find a global solution in a complex environment with local minima. Therefore, to integrate a whole-body controller with a task executer, a planner is required that
provides connectivity information at a global scope while constraining only the DoFs that are relevant for the (sub-task) at hand.
The contribution of this work is to apply a topological planner for manipulation, where the topological graph is an abstract representation of possible robot configurations that describes possible sequences of motions.
All nodes in the topological graph are grounded to specific end-effector attractors. Each attractor can constrain up to six DoFs and possesses a number of properties, such as the target pose, target offset, stiffness and goal tolerances. The feasibility of the motion is subsequently validated by forward integration of the closed-loop whole-body controller.

\subsubsection{Results}

\paragraph{Reports}
The report is a result of the work performed by multiple people at the TUE; contributions were made by Lunenburg, J.J.M., Derksen T.J.J. and Metsemakers, P.M.G.. Next to this deliverable report, the following reports can be consulted on whole-body-motion control for general purpose service robots.
\begin{longtable}{|p{8.5cm}p{4cm}p{2cm}|}
\hline
\rowcolor[gray]{0.8} \bf Title / URL & \bf Author & \bf Year \\
\hline
\href{http://repository.tue.nl/762360}{Reactive collision avoidance for domestic service robots} & Metsemakers, P.M.G. & 2013 \\
\href{http://repository.tue.nl/778410}{Global path planning for the reactive whole-body controller} & Derksen T.J.J & 2014 \\
\href{http://repository.tue.nl/794842}{Context-aware design and motion planning for autonomous service robots} & Lunenburg, J.J.M. & 2015 \\
\hline
\end{longtable}

\paragraph{Software}
While doing research to the topic of whole-body motion planning, two open-source software packages are developed. These ROS-packages are publicly available on \href{http://github.com}{github.com}. Section \ref{ap:integration} describes how the two different packages can be integrated in a third party software stack.
\begin{longtable}{|p{6cm}p{9cm}|}
\hline
\rowcolor[gray]{0.8} \bf Package / URL & \bf Description\\
\hline
\href{https://github.com/tue-robotics/amigo_whole_body_controller}{tue-robotics/whole\_body\_controller} & Whole-Body Controller ROS Package \\
\href{https://github.com/tue-robotics/whole_body_planner}{tue-robotics/whole\_body\_planner} & Whole-Body Cartesian planning utility ROS Package \\
\hline
\end{longtable}

\paragraph{Videos}
\label{videos}
The following videos demonstrate the developed whole-body controller and planner derived from the work within this project. 
\begin{longtable}{|p{6cm}p{9cm}|}
\hline
\rowcolor[gray]{0.8} \bf Video / URL & \bf Description\\
\hline
\href{https://youtu.be/7GcLU9l65eM}{youtube/reactive\_collision\_avoidance} & Reactive collision avoidance with the AMIGO robot \\
\href{https://youtu.be/dibi1lapwOE}{youtube/whole\_body\_planning} & Whole-Body motion planning with the AMIGO robot\\
\hline
\end{longtable}

